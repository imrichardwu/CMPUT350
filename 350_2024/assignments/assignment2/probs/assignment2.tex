\documentclass[a4paper,11pt]{article}
\include{../../../latex_headers/assignment_header}


% ---------------------------------- Document ----------------------------------
\begin{document}
\pagenumbering{arabic}

\begin{center}
{\Large CMPUT 350 Assignment 2} \\
{Due: Wednesday Oct. 9, 22:00}
\end{center}

\linerule

\textbf{Important}:
Read this text completely and start working on the assignment as
soon as possible. Only then will you get an idea about how long it will take
you to complete it. 

\medskip

If you worked on the assignment on your personal computer, copy your files to
the undergraduate lab machine (e.g. uf13.cs.ualberta.ca) with
scp and then test your code there as well before submitting it. 
Like the labs, we will be using this environment to test your code.

\medskip

When done, submit your solution files on eClass:

\begin{center}
    \texttt{README.txt World.cpp Marine.cpp Tank.cpp Experiments.txt Problem2.txt/pdf}
\end{center}


\bigskip 

Your submissions will be marked considering their correctness (syntax and
semantics), documentation, efficiency, and consistent coding style. Add assert
statements to check important pre/post-conditions.

\medskip

Finally, make sure that you follow the course collaboration policy which is
stated on the course webpage. File \texttt{README.txt} must be completed!

\linerule

1. RTS-Combat Simulator [116 marks] \\
In this part you will complete a simulator for an RTS-like combat game.
Circular units representing marines and tanks are roaming a rectangular map.
They fight opponents' units whenever they come in attack range, but their
motion policy is simplistic: they never collide with other units, but when
hitting the map border, they either stop or bounce off similar to billiard
balls.

\bigskip

We provide executable file \texttt{simul.lab} which runs on the lab machines and shows
you how the simulator is supposed to work (in principle) when all parts have
been implemented. Because there are some implementation choices, we don't
expect your program to work exactly as \texttt{simul.lab}

\bigskip

The makefile can generate code for 2 scenarios: Option 1 creates code that can
be run remotely through an \texttt{ssh -X ... session} (provided your local computer
runs an XWindows server). Option 2 is generates a (faster) version that only
runs when sitting in front of a lab machine. Comment out the option you
don't use.

\bigskip

The simulator takes various parameters. To see a list, run \texttt{./simul.lab x}. To
check your code for memory leaks run valgrind with \texttt{--leak-check=full}. This
will report where leaks exactly happen. This way you can distinguish leaks in
GL and glut from yours. Another idea is to test your code without graphical
output to speed up experiments.

\bigskip

To compile the project, run make. This creates executable file simul, which by
default opens an OpenGL window showing hundreds of units fighting until one
team has no units left (or both).

\bigskip

We consider two unit types: Marine and Tank. They have the following 
unit stats:
\begin{verbatim}
                       Marine     Tank
    radius              10         20
    attack_radius       40         80
    max_speed           10         15
    damage               1          4
    hp                  45        100
\end{verbatim}

Those properties have to be initialized in the units' constructors.
There are two steps of solving this problem:
\begin{enumerate}
    \item 
    Understand the provided code and complete it by implementing all functions
    commented with \texttt{// ... implement} 

    \medskip
 
    Only change the source files you will submit: \texttt{World.cpp} \texttt{Marine.cpp} \texttt{Tank.cpp} .
    We won't see any changes you make to \texttt{.h} files and to \texttt{World2.cpp}

    \medskip
 
    We suggest to compare the output of your simulator with that of \texttt{simul.lab}'s
    using small unit counts and bounce switched off and on

    \medskip
 
    If you encounter runtime issues, it is a good idea to pass on a random
    number seed != 0 to make program runs repeatable

    \medskip

    \item 
    Use YOUR program in a set of experiments that determine which of the three
    attack policies (selecting a random weakest, random closest, and random
    most dangerous target) is the strongest for the following parameter
    settings:
 
\begin{verbatim}
    width=height    :   700
    marines=tanks   :   20, 50, 100, 300
    bounce          :   true
    redpol/bluepol  :   all combinations
\end{verbatim}
 
    To gain statistical confidence, run each setting 100+ times (with differnt
    seeds). We'd like to see a brief writeup in \texttt{Experiment.txt} explaining your
    experiments and how you interpret the results
\end{enumerate}

For this programming assignment you CAN use {\CC} containers or functions from
\texttt{STL} and Boost that are installed on the lab machines. In particular, class
template \texttt{std::vector<T>} will be used extensively in this assignment which we
will cover later in the course. For our purposes, you only need to know that a
\texttt{std::vector<T> v;} behaves like a dynamic array of type \texttt{T}. E.g., \texttt{std::vector<int> v;}
defines an empty \texttt{int} vector. You can access elements by index like arrays
(e.g., \texttt{v[0]}), you can query its size by calling \texttt{v.size()}, you can add elements
at the end using \texttt{v.push\_back(x);} and remove the last element using
\texttt{v.pop\_back()}. If you need more information about \texttt{std::vector}, please consult
\url{http://www.cplusplus.com/reference/vector/vector/} or other online material.

\bigskip 

Your submission must compile on the lab machines without errors/warnings using
the provided makefile and running ``\texttt{make}'' . It will be marked considering its
correctness (syntax + semantics), documentation, efficiency, and coding
style. Add assert statements to check important pre/post-conditions

\linerule

\bigskip

2. [40 marks] 
For this part please provide solutions (i.e., formal proofs/disproofs) in file
\texttt{Problem2.txt} (or \texttt{Problem2.pdf} if you prefer to use a text processor like \LaTeX \ 
or Word). Be as formal as you can, demonstrating - based on the definitions
introduced in class - a rigorous chain of logical implications leading from
the premise to the conclusion - or providing a counter example:

\begin{enumerate}[label=(\alph*)]
    \item Prove that optimal heuristics (i.e. $h^*(n)$) are consistent
    \item Prove: If $h_1(n), \dots, h_k(n)$ are admissible, so is $h(n) = \text{max}(h_1(n), \dots, h_k(n))$
    \item Prove: If $h_1(n), \dots, h_k(n)$ are consistent, so is $h(n) = \text{max}(h_1(n), \dots, h_k(n))$
    \item Prove or disprove: If $h_1(n), \dots, h_k(n)$ are admissible, so is $h(n) = h_1(n) + ... + h_k(n)$
    \item Prove or disprove: If $h_1(n), \dots, h_k(n)$ are admissible, so is $h(n) = (h_1(n) + ... + h_k(n))/k$
\end{enumerate}

\end{document}