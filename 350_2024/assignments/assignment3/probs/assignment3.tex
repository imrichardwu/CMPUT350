\documentclass[a4paper,11pt]{article}
\include{../../../latex_headers/assignment_header}


% ---------------------------------- Document ----------------------------------
\begin{document}
\pagenumbering{arabic}

\begin{center}
{\Large CMPUT 350 Assignment 3} \\
{Due: Wednesday Oct. 23, 22:00}
\end{center}

\linerule

\textbf{Important}:
Read this text completely and start working on the assignment as
soon as possible. Only then will you get an idea about how long it will take
you to complete it. 

\medskip

If you worked on the assignment on your personal computer, copy your files to
the undergraduate lab machine (e.g. uf13.cs.ualberta.ca) with
scp and then test your code there as well before submitting it. 
Like the labs, we will be using this environment to test your code.

\medskip

When done, submit your solution files on eClass:

\begin{center}
    \texttt{README.txt GridPriv.h GridInclude.h Grid.cpp Problem2.txt/pdf}
\end{center}


\bigskip 

Your submissions will be marked considering their correctness (syntax and
semantics), documentation, efficiency, and consistent coding style. Add assert
statements to check important pre/post-conditions.

\medskip

Finally, make sure that you follow the course collaboration policy which is
stated on the course webpage. File \texttt{README.txt} must be completed!

\linerule

1. [24 marks] \\
The objective of this assignment part is to produce a class that can store a
map-like structure and perform pathfinding and connectivity operations on it.

\bigskip

Class \texttt{Grid} represents a rectangular tile map with octile topology, i.e. 8
compass directions using Euclidean distance. There are blocked, ground, or
water tiles. Moving objects are not represented in the grid.

\bigskip

The moving object for pathfinding has a location $(x,y)$ and size $s$. It is
considered to occupy the tiles from $x$ to $x+s$ in the x-axis, and from $y$ to $y+s$
in the y-axis. Thus, an object with size 0 occupies one tile, and object with
size 1 occupies a 2x2 square, and an object with size 2 occupies a 3x3
square. You are only required to support objects of sizes 0, 1, and 2.

\bigskip

Moves are permitted if and only if all tiles passed through during the move
have the same tile type. For instance, if a 3x3 object is on ground tiles, it
may only move diagonally if the 4x4 region it passes over is composed solely
of ground tiles.

Similarly, a 2x2 object on water tiles may move to the east if the 3x2 region
it passes over is composed entirely of water tiles.

\bigskip 

Example:
\begin{verbatim}
    01234
  0 wwggg   A 3x3 object centered in this 5x5 map may move west, south, or
  1 ggggg   southwest, because it will pass only over ground tiles. However,
  2 ggggg   it may NOT move northeast, even though it would wind up on a 3x3
  3 ggggw   patch of ground tiles, because it would clip through some water
  4 ggggw   tiles when moving
\end{verbatim}

Note that in the above example, the object's initial location would be $(1,1)$
and its size would be 2, meaning it occupies from 1 to 3 on both the $x$ and $y$
axises.

\bigskip

To test your implementation issue make which creates \texttt{testGrid}. See
\texttt{TestGrid.cpp} for documentation.
\texttt{exampleGrid} is our solution that should run on lab machines.

\bigskip

Do not include a \texttt{main()} function in your submitted code. 
If you want to test your function separately, define
your \texttt{main()} function in a separate \texttt{.cpp} file and link with the corresponding
\texttt{.o} file.

\bigskip 

Also, private class members you might want to add to class \texttt{Grid} must be
defined in file \texttt{GridPriv.h}, which you will submit together with \texttt{Grid.cpp}.

\bigskip 

You may use any C/{\CC} standard libraries you
want, such as STL, and Boost. Feel free to use {\CC}17 features that \texttt{g++} on the
lab computers support. We will compile your code using \texttt{-std=c++17}.

\linerule

\bigskip

2. PRA* Map Abstraction [16 marks] 

\begin{enumerate}[label=(\alph*)]
\item 
    Apply the clique-based abstraction used in PRA* (\textbf{ai-part2, page 12}) to the
    graph below.
    For this, scan the nodes from \textbf{top-left} to \textbf{bottom-right} - row by
    row, identifying new maximum size complete subgraphs of size \textbf{up to 4} that can
    be abstracted into new nodes in the next abstraction level - starting from
    the current node and scanning its unvisited neighbourhood. Continue this
    process until only one node remains, showing the resulting abstraction levels
    and indicating which nodes got abstracted by what node in the process. 
    \textbf{Use uppercase letters A,B,C,...} 
    to name new abstract nodes and explain all steps.
{\Large
\begin{verbatim}
    a-b-c-d-e-f-g
    |   |   |   |
    h-i j-k l-m-n 
    |   |       |
    o-p-q-r-s-t-u
\end{verbatim}
}

\item 
    Assuming the original graph size is $N=4^n$ and clique-based
    abstraction being able to \textbf{always group 4 nodes} to form a new abstract node in
    each step:
    \begin{enumerate}[label=(\roman*)]
        \item how many nodes does the $k$-th abstraction level consist of ($k=0$: original
        graph with $4^n$ nodes, $k=1$: first abstraction level with ? nodes, etc.)?
        \item which abstraction level contains exactly one node? Explain
        \item assuming we reach one node at abstraction level $n=2u$ for an integer $u$,
        how many nodes does abstraction level $u$ consist of in relation to the
        original number $N=4^n$? Simplify your answer and explain your steps
        \item Prove that the clique abstraction used in PRA* preserves
        connectivity, i.e., given $A$,$A'$ and $B$,$B'$, where $A'$ is the abstract node for
        $A$ and $B'$ is the abstract node for $B$, there is a path from $A$ to $B$ in the
        original map if and only if there is a path from $A'$ to $B'$ in the abstract
        map. \textbf{Note}: your proof needs to show both implication directions.
    \end{enumerate}
\end{enumerate}

\end{document}