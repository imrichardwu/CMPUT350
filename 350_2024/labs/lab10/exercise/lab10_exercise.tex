\documentclass[a4paper,11pt]{article}
\include{../../../latex_headers/lab_header}


% ---------------------------------- Document ----------------------------------
\begin{document}
\pagenumbering{arabic}

\begin{center}
{\Large CMPUT 350 Lab 10 Exercise Problems}
\end{center}

\labrules{g++ -Wall -Wextra -O -g -std=c++17 ...}{Solve.cpp}{10}


In this exercise you will complete the matrix game tool you worked on in the
today's prep phase.
Start by downloading the files provided.

\linerule 

1. [8 marks]
In file \texttt{Solve.cpp}, implement function
\begin{cppcode*}{linenos=false}
void best_response_to_col(const Matrix &A, const Vector &strat2);
\end{cppcode*}
which gets invoked when using option -c and is analogous to the
\texttt{best\_response\_to\_row()} you implemented in the prep phase.

\medskip

Instead of finding a best reponse to a row player strategy, you now need to
compute and print a best response strategy to a given column player strategy,
together with the row player's value.

\medskip

Test your implementation by comparing its output to that of \texttt{./solve -c < ...}
with various input files (\texttt{*.mgc} and some you create).

\linerule 

\newpage 

2. [10 marks]
In file Solve.cpp, implement function
\begin{cppcode*}{linenos=false}
void solve(Matrix &A);
\end{cppcode*}
which gets invoked when using option -s.

It reads a payoff matrix for the row player from \texttt{stdin} and computes MiniMax
strategies for the row and column players and the game value for the row
player, and writes the results to \texttt{stdout}.

\medskip

Strategies are computed by writing an LP to a file, then running external
program \texttt{lp\_solve} on that file, and finally extracting results from the output
it generates.

\medskip

Your task is to write LP representations created from the input payoff matrix
to the file \texttt{lp\_solve} reads, using the LPs for solving matrix games developed
in the notes (AI Part 5, pp.25-26).
See the comments in \texttt{solve()} for details.

\medskip

\textbf{HINT}: Test your implementation by comparing its output to that of \texttt{./solve -s < ...}
with various input files (\texttt{*.mgs} and some you create).
Run \texttt{./solve -s < test2x4.mgs} and look at \texttt{p1.lp}, etc. for the expected format of 
how to write to these files in your solution code.

\medskip

For example, for input \texttt{RPS.mgs} :
\small{
\begin{verbatim}
    3 3
    0 -1 +1
    +1 0 -1
    -1 +1 0
\end{verbatim}
}

the output of \texttt{./solve -s < RPS.mgs} is
\small{
\begin{verbatim}
    3 by 3 game
    +0.000000 -1.000000 +1.000000 
    +1.000000 +0.000000 -1.000000 
    -1.000000 +1.000000 +0.000000 
    value for row player p1: 0
    strategy p1: 0.333333 0.333333 0.333333
    strategy p2: 0.333333 0.333333 0.333333
\end{verbatim}
}
\end{document}