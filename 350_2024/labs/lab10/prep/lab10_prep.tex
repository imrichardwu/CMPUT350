\documentclass[a4paper,11pt]{article}
\include{../../../latex_headers/lab_header}


% ---------------------------------- Document ----------------------------------
\begin{document}
\pagenumbering{arabic}

\begin{center}
{\Large CMPUT 350 Lab 10 Prep Problems}
\end{center}

% Add important dates, change TA, add TA tools

\linerule

Contained are the following files:
\begin{itemize}
    \item \texttt{Solve.h}
    \item \texttt{Solve.cpp}
    \item \texttt{solve\_main.cpp}
    \item \texttt{make}(a shell script for generating a.out)        
    \item \texttt{solve} (working Linux executable - for reference, 
        use \texttt{chmod +x solve} to make it executable)
    \item \texttt{RPS.mge} (test input for -e option - Rock-Paper-Scissors!)
    \item \texttt{RPS.mgr} (test input for -r option)
    \item \texttt{test2x4.mge} (test input for -e option)
    \item \texttt{test2x4.mgr} (test input for -r option)
\end{itemize}

In this lab you will implement a software tool for processing zero-sum matrix games.
Data will be read from \texttt{stdin}, and results are printed to \texttt{stdout}. \\

We provide a working executable (\texttt{solve}) and input samples that can
be used to check whether your program generates correct output, like so:
{\small 
\begin{verbatim}
    ./solve -e < RPS.mge
    ./a.out -e < RPS.mge      # should generate same output
    ./solve -r < test2x4.mgr
    ./a.out -r < test2x4.mgr  # should generate same output
\end{verbatim}
}

The input consists of an N1 by N2 payoff matrix which is encoded like so:
{\small 
\begin{verbatim}
    3 3
    0 -1 +1
    +1 0 -1
    -1 +1 0
\end{verbatim}
}
(i.e., N1=\#rows and N2=\#cols followed by (N1*N2) payoff values for the row player) \\

Depending on the command line option (-r or -e), one or two
vectors of length N1 or N2 follow the matrix in the input (see below).

\linerule

1. In file \texttt{Solve.cpp} implement function
\begin{cppcode*}{linenos=false}
void expected_value(const Matrix &A, const Vector &strat1, const Vector &strat2);
\end{cppcode*}
which is invoked when using the -e option. 


It takes a payoff matrix (in view of the row player), a row player strategy,
and a column player strategy as input and computes the expected game result
for the row player (see AI Part 5 p.20) and writes it to \texttt{stdout}.

For example, for input file \texttt{RPS.mge} which contains the standard
Rock-Paper-Scissors payoff matrix, a row strategy, and a column strategy:
{\small 
\begin{verbatim}
    3 3
    0 -1 +1
    +1 0 -1
    -1 +1 0
    0.2  0.2 0.6
    0.25 0.5 0.25
\end{verbatim}
}

the output of \texttt{./solve -e < RPS.mge} is:
{\small 
\begin{verbatim}
    3 by 3 game
    +0.000000 -1.000000 +1.000000 
    +1.000000 +0.000000 -1.000000 
    -1.000000 +1.000000 +0.000000 
    row strategy: 0.2 0.2 0.6
    col strategy: 0.25 0.5 0.25
    expected value for p1: 0.1
\end{verbatim}
}

Test your program by comparing its output to that of \texttt{./solve -e < ...} for
various input files (including \texttt{RPS.mge}, \texttt{test2x4\_mge}, and some that you create).

\linerule

2. In file \texttt{Solve.cpp} implement function
\begin{cppcode*}{linenos=false}
void best_response_to_row(const Matrix &A, const Vector &strat1);
\end{cppcode*}
which is invoked when using the -r option.

It takes a payoff matrix (in view of the row player) and a row player
strategy as input and computes a best-response strategy for the column
player and the resulting game value for the row player and writes them \texttt{stdout} (see AI
Part 5 p.23).

For example, for input file \texttt{RPS.mgr} which contains the standard
Rock-Paper-Scissors payoff matrix and a row strategy:
{\small 
\begin{verbatim}
    3 3
    0 -1 +1
    +1 0 -1
    -1 +1 0
    
    0.2 0.2 0.6
\end{verbatim}
}

the output of \texttt{./solve -r < RPS.mgr} is:
{\small 
\begin{verbatim}
    3 by 3 game
    +0.000000 -1.000000 +1.000000 
    +1.000000 +0.000000 -1.000000 
    -1.000000 +1.000000 +0.000000 
    row strategy: 0.2 0.2 0.6
    best response to row strategy: 1 0 0
    value for p1: -0.4
\end{verbatim}
}

I.e., a best response to the row strategy that chooses Rock 20\%, Paper 20\%,
and Scissors 60\% of the time is the 100\% Rock column player strategy.

Test your program by comparing its output to that of \texttt{./solve -r < ...} for
various input files (including \texttt{RPS.mgr}, \texttt{test2x4.mgr}, and some that you create).

\end{document}