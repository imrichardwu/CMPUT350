\documentclass[a4paper,11pt]{article}
\include{../../../latex_headers/lab_header}



% ---------------------------------- Document ----------------------------------
\begin{document}
\pagenumbering{arabic}

\begin{center}
{\Large CMPUT 350 Lab 7 Prep Problems}
\end{center}

% Add important dates, change TA, add TA tools

\linerule

1. Write function template \texttt{selection\_sort} that takes an iterator range and
a comparison functor as parameters and sorts the range according to the
functor using selection sort (i.e., repeatedly choosing the minimum of the
remaining values and swapping it with the current element), so that the 
following code works:

\begin{cppcode*}{linenos=false}
...
struct MyLess { ... };
struct MyGreater { ... };
vector<int> v{3,5,4};       // vector contains 3 5 4
selection_sort(begin(v), end(v), MyLess());         // 3 4 5
selection_sort(begin(v), end(v), MyGreater());      // 5 4 3
\end{cppcode*}

\linerule 

2. Write a program that reads integers from \texttt{stdin} and prints all encountered
values exactly once in non-decreasing order to \texttt{stdout}.

\[ \texttt{E.g.: 5 3 4 1 4 3 => 1 3 4 5} \]

Can you give a non-trivial upper bound on the runtime of your algorithm
depending on the number of ints in the input (n) ?

\medskip

\textbf{Hint}: in what order are elements visited when traversing a \texttt{std::set<int>} ?

\linerule 

3. 

\begin{enumerate}[label=\alph*)]
    \item Write function \texttt{long long choose(int n, int k)} that for $k \ge 0$ and $k \le n$ computes
        \begin{itemize}
            \item $\text{choose}(n, k) = 1$, for $k=0$ or$ k=n$
            \item $\text{choose}(n, k) = \text{choose}(n-1,k-1) + \text{choose}(n-1, k)$, otherwise
        \end{itemize}
        
        [also known as binomial coefficient -
        \[ \binom{n}{k} \]
        or the number of ways of choosing k objects out of n; above recursion is known
        as ``\textit{Pascal's Triangle}'']

    \item Write function \texttt{choose\_m} that speeds up computing \texttt{choose(n,k)} by memoization
        (i.e., storing and recalling already computed function values in a \texttt{std::map},
        see lecture cpp notes for an example)

    \item Lastly, determine the highest value of \texttt{n} for which \texttt{choose\_m(2*n, n)} computes
    the correct value. For this, consider using Boost's arbitrary precision
    integer type \texttt{cpp\_int} like so:
    \end{enumerate}
\begin{cppcode*}{linenos=false}
#include <boost/multiprecision/cpp_int.hpp>
using namespace boost::multiprecision;
...
// look - HUGE numbers!
cpp_int i = 2;
cpp_int j("100000000000000000000000000");
cout << i*j << endl;   // prints 200000000000000000000000000     
\end{cppcode*}


\end{document}