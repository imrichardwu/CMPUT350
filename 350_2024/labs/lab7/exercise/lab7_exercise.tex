\documentclass[a4paper,11pt]{article}
\include{../../../latex_headers/lab_header}



% ---------------------------------- Document ----------------------------------
\begin{document}
\pagenumbering{arabic}

\begin{center}
{\Large CMPUT 350 Lab 7 Exercise Problems}
\end{center}

\labrules{g++ -Wall -Wextra -Wconversion -Wsign-conversion -O -g -std=c++17 ...}{Histo.h histoMain.cpp Sorter.h sorterMain.cpp}{7}


1. Histogram [25 marks]

Implement a program that reads words from \texttt{stdin} and keeps track of word counts
and when done prints the histogram in non-increasing count order like so:

{\small
\begin{verbatim}
    the 392
    moon 10
    bar 1
    clown 1
\end{verbatim}
}

In case of identical counts the lexicographically smaller word is printed
first (e.g., bar before clown above)

\medskip

For this purpose, define class \texttt{Histo} in file \texttt{Histo.h} that has the following
public methods:
\begin{cppcode*}{fontsize=\footnotesize,linenos=false}
// increment count for word w
void count(const std::string &w) {
    ... implement
}

// print histogram in non-increasing count order to stream os
void print(std::ostream &os) const {
    ... implement
}
\end{cppcode*}

Implement these functions on the spot (i.e., there is no \texttt{Histo.cpp} file). Make
use of STL containers and algorithms (\texttt{map}, \texttt{vector}, \texttt{sort} ...) as much as
possible. If unsure about STL data types and algorithms consult your lecture
notes or the official C/{\CC} reference page.

\medskip 

In file \texttt{histoMain.cpp} implement function \texttt{main()}, which uses \texttt{Histo} for the
program's task described above. You can assume that each word occurs at most 2
billion times.

\medskip 

Test it thoroughly and make sure your program doesn't leak memory and is well
documented.

\medskip 

\textbf{Hints}:
\begin{itemize}
    \item \texttt{std::string s; cin >> s;} reads the next word into string variable \texttt{s}
    \item consider range-based for loops and \texttt{auto}, or \texttt{cbegin()}, \texttt{cend()} when iterating
    through containers in constant member functions
    \item sort doesn't work with associative containers because they are sorted implicitly
\end{itemize}

{\small
\begin{verbatim}
    Test input:

    c b a
    zz zz xx xx zz xx
    
    Test output:
    
    xx 3
    zz 3
    a 1
    b 1
    c 1
\end{verbatim}
}


\linerule 

2. Sorter [15 marks]


For this question you will produce a simple class Sorter in file \texttt{Sorter.h} that
can sort data by different criteria. It must accept some number of points
given by two coordinates of type \texttt{double} each, specified by \texttt{add\_point(x, y)},
and store them. Then, when \texttt{print\_sorted\_closest\_to(x, y)} is called, it must
sort the points based on their distance to the specified coordinates, and
write them to \texttt{stdout} - one point per line - from closest to
farthest. Implement all methods of \texttt{Sorter} in \texttt{Sorter.h}

\medskip 

To test class Sorter write a program in \texttt{sorterMain.cpp} that reads $n > 0$ points
given by 2 \texttt{double} values each from \texttt{stdin} and prints the last $n-1$ points in
order closest to farthest from the first point.

\medskip 

Make sure your program doesn't leak memory and is well documented. Test it
thoroughly.


\textbf{Hints}:
\begin{itemize}
    \item The square root function sqrt is defined in header file cmath. Think
        about a way around using sqrt, because it is slow
    \item Consider using a functor object that stores the first point
\end{itemize}

\pagebreak

{\small
\begin{verbatim}
    Test input:

    1 1
    3 3 
    2 2 
    
    Test output:
    
    2 2
    3 3
    
    ===========
    
    Test input:
    
    1 1
    2 2 
    2 2 
    
    Test output:
    
    2 2
    2 2
\end{verbatim}
}

\end{document}