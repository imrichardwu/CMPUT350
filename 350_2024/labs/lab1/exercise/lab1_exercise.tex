\documentclass[a4paper,11pt]{article}
\include{../../../latex_headers/lab_header}


% ---------------------------------- Document ----------------------------------
\begin{document}
\pagenumbering{arabic}

\begin{center}
{\Large CMPUT 350 Lab 1 Exercise Problems}
\end{center}

\labrules{g++ -g -Wall -Wextra -Wconversion -Wsign-conversion -O -std=c++17 ...}{p1.cpp p2.cpp}{1}

1. [9 marks] Write {\CC} program \texttt{p1.cpp} which implements and tests function
\[ \cppinline{void rotate_right(int A[], int n);} \]

\texttt{rotate\_right} shifts all elements of n-element array \texttt{A} one location to the right, 
and stores the original right-most entry into the left-most position. \\

\textbf{Examples}:
\begin{itemize}
    \item \texttt{A[]} before call:    1 2 3 4 5 6
    \item \texttt{A[]} after call:     6 1 2 3 4 5
    \item after another call: 5 6 1 2 3 4
    \item etc.
\end{itemize}

Test your function using a couple of test cases in \texttt{main()} including one which
rotates an n-element array \texttt{n} times and checks whether the resulting array is
the same as the one you started with.
 
\linerule

\newpage

2. [12 marks] Write {\CC} program \texttt{p2.cpp} that reads a sequence of int array pairs 
(in decimal notation with white-space characters [space, newline, tab] as delimiter) from
\texttt{stdin} and prints their component-wise product to stdout. \\

Inputs consist of multiple instances. Your program must handle them all and
only stop when encountering the end of the input or an input error. 
You may assume that each individual instance fits in memory. 
In case of an input error, your program \textbf{needs} to report ``\texttt{input error}'' to
\texttt{stderr} and \textbf{return with value 1}. 
If everything is OK, the program needs to \textbf{return with value 0}. \\

\textbf{Example}: Suppose \texttt{input.txt} contains the following 3 instances:
\begin{cppcode}
10
0 1 2 3 4 5 6 7 8 9
9 8 7 6 5 4 3 2 1 0
5
2 3 4 -1 6
1 1 1 1 1
1
8
5
\end{cppcode}

Running the command \texttt{./p2 < input.txt} should then yield the following output:
\begin{verbatim}
10
0 8 14 18 20 20 18 14 8 0
5
2 3 4 -1 6
1
40
\end{verbatim}

In general, the input looks as follows:
\begin{verbatim}
  Instance_1
  Instance_2
  ...
  Instance_k
\end{verbatim}

for which each instance is described by:
\begin{verbatim}
  n            # (> 0) length of the two input arrays
  n ints       # first array
  n ints       # second array
\end{verbatim}

The output for each instance has the form:
\begin{verbatim}
  n            # length of output array
  n ints       # output array (= product of two input arrays)
\end{verbatim}

Handle each input instance one at a time, 
and outputting the result before moving onto the next instance.

\newpage

\textbf{Hints}:
\begin{itemize}
    \item All you need for input is \texttt{std::cin >> input}; where input is an int variable
        (the line format is irrelevant, what matters is that all numbers are
        separated by white-space characters, which \texttt{cin >> input} consumes but ignores)
    \item For testing, prepare text files with various inputs (legal and illegal) and
        redirect them into your application like so: \texttt{./a.out < test-input}
    \item Use \texttt{valgrind} to check for memory leaks (see Lab 0)
\end{itemize}



\end{document}