\documentclass[a4paper,11pt]{article}
\include{../../../latex_headers/lab_header}

% ---------------------------------- Document ----------------------------------
\begin{document}
\pagenumbering{arabic}

\begin{center}
{\Large CMPUT 350 - C Refresher}
\end{center}

% ------------------------------------------------------------------------------

\section{Hello World}
Create file \texttt{hello.c} containing this program (copy \& paste):
\begin{cppcode}
// this program prints "hello world" to standard output
// let the compiler know about standard I/O functions
#include <stdio.h>
int main() {
    printf("hello world\n");
    return 0;
}
\end{cppcode}
Then issue 
\[ \texttt{gcc -o hello hello.c} \]
which generates executable file \texttt{./hello} from source file \texttt{hello.c}.
The program prints "hello world" when being invoked by issuing \texttt{./hello<ENTER>}.
\texttt{gcc} is the C compiler in the GNU Compiler Collection. \\

In CMPUT 350 we will be using its cousin g++ for compiling C++ programs.
For projects involving StarCraft you will have to install Windows and Visual.
C++ on your computer (or in a virtual machine if you use Linux or Mac OS).


% ------------------------------------------------------------------------------
\section{Important \texttt{gcc}/\texttt{g++} command line paramters}
\begin{itemize}
    \item -O, -O2, -O3 : optimize executable code at (-O3 fastest)
    \item -Wall -Wextra -W -Wundef -Wconversion -Wsign-conversion: switch on useful compiler warnings
    \item -g: generate debug information (for debugger gdb)
\end{itemize}

To learn more about command line options issue: \texttt{man g++} or \texttt{man gcc}.

% ------------------------------------------------------------------------------
\section{C Types and Variables}
\begin{itemize}
    \item basic: \cppinline{char, short, int, long int, float, double}
    \item arrays:
\begin{cppcode}
int foo[100];       // defines array foo that contains 100 ints
                    // on the stack, valid indicies are 0.99
int x = foo[42];    // stores array element with index 42 into variable x
\end{cppcode}
    \item pointers:
\begin{cppcode}
int x = 5;  // defines int x (on stack)
int *px = &x;   // address of x is assigned to int pointer px
                // px no points to x 
*px = 0;        // px points to x, *px therefore refers to x.
                // This sets x to 0
                // & is the address of operator, * is the pointer dereference operator
\end{cppcode}
    \item structures: Store data items together using a common name. Access by component name:
\begin{cppcode}
struct X {
    int a, b;           // structure X with components a,b,c,d
    float c, d;
};

struct X x;             // variable definition (x has type X)
x.a = 5;                // assign 5 to structure component a
printf("%d", x.b);      // print x's b component to stdout
\end{cppcode}
\end{itemize}

% ------------------------------------------------------------------------------
\section{Control Flow}
\begin{itemize}
    \item Conditionals:
\begin{cppcode}
if (x > 0) {
    // executed if x > 0
} else if (x < 0) {     // [optional]
    // executed if x < 0
} 
else {                  // [optional]
    // executed if x == 0
} 
\end{cppcode}
    \item Loops:
\begin{cppcode}
while (x > 0) {     // execute as long as x > 0
    --x;            // decrease x by 1
}

int i;
for (i=0; i < n; ++i) {     // execute for i = 0 .. n-1
    printf("%d\n", i);
}

// is equivalent to:
int i = 0;
while (i < n) {
    printf("%d\n", i);
    ++i;
}
\end{cppcode}
    \item Functions: parameters are passed by value (except for arrays, see below)
\begin{cppcode}
// function that returns an int, takes int parameter x, and returns x+1
int foo(int x) { return x + 1; }

int y = foo(5); // y = 6 after executing this line

// function that takes int array and number of elements
// as parameter and returns sum of array elements
int sum(int A[], int n) {
    int sum = A[0];
    int i;  // in C++ can be moved inside the for loop
    for (i=1; i < n; ++i) {
        sum += A[i];
    }
    return sum;
}
\end{cppcode}
\end{itemize} 

In C, arrays are passed by reference. 
What is actually passed is just a pointer to the first element. 
Consequently, array parameters don't know the size of the array. 
So, the size has to be passed separately. \\

Equivalent function definition (but harder to understand: is it a pointer we pass or an array?)
\begin{cppcode}
int sum(int *A, int n) {
    // ...
}
\end{cppcode}


\section{Memory Allocation}
\texttt{malloc} allocates m consecutive bytes on heap (requires \texttt{stdlib.h}).
Since \texttt{malloc} returns a \texttt{void*}, it needs to be cast to the appropriate type:
\[ \texttt{int *p = (int*) malloc(n * sizeof(int));} \]
allocates array of n ints on heap and assigns address of first int to \texttt{p}. 
\texttt{(int*)} is a type cast that convinces the compiler that the returned value is indeed a pointer to int.\\

\texttt{free(p)} returns memory \texttt{p} to the operating system.
Using \texttt{*p} after this call is a logical error.


\end{document}