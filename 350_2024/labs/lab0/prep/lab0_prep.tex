\documentclass[a4paper,11pt]{article}
\include{../../../latex_headers/lab_header}

% ---------------------------------- Document ----------------------------------
\begin{document}
\pagenumbering{arabic}

\begin{center}
{\Large CMPUT 350 Lab 0 Prep Problems}
\end{center}

\linerule

1. Define a C function \texttt{array\_add} that adds an integer array to another element-wise. \\

Type the following code into file \texttt{ex1.c} and then add function
\texttt{array\_add} with your favourite editor. 
Also print array \texttt{u} element by element after calling \texttt{array\_add}. 
Then compile the program using \texttt{gcc -Wall -Wextra ex1.c}. 
This should not generate any warnings or errors. \\

When running your program, can you explain the output you see?

\begin{cppcode*}{linenos=false}
#define N 20   // C-style constant: N gets replaced by 20 below
               // (deprecated in C++)
int main() {
    int u[N]; // allocates int arrays of length N on stack
    int v[N];
    // adds v[i] to u[i] for i=0..N-1
    array_add(u, v, N);
    return 0;
}
\end{cppcode*}
 
\linerule

2. Consider this Matrix structure which describes a matrix comprised of rows*cols
double numbers:

\begin{cppcode*}{linenos=false}
struct Matrix {
    int rows;
    int cols;
    double *a;  // pointer to rows*cols elements
};
\end{cppcode*}

Implement these functions in \texttt{ex2.c}:
\begin{cppcode*}{linenos=false}
// initialize matrix pointed to by m with r rows and c columns
// i.e. allocate sufficient memory and set all elements to 0
void init(struct Matrix *m, int r, int c);

// free memory associated with matrix pointed to by m
void deallocate(struct Matrix *m);
\end{cppcode*}

Usage:

\begin{cppcode*}{linenos=false}
int main() {
    struct Matrix m;
    init(&m, 20, 30);  // pass address of m and dimensions to function
    // ... use matrix m
    deallocate(&m);
    return 0;
}
\end{cppcode*}

Start by copying and pasting above code snippets into file \texttt{ex2.c}.
Compile your program with:
\[\texttt{gcc -Wall -Wextra ex2.c}\]
and make sure that there are no errors or warnings.


\linerule

3. Copy \texttt{ex1.c} to \texttt{ex1buggy.c}, 
add a segfault bug 
(such as \texttt{int *p = 0; *p = 0} in main),
and compile it with \texttt{gcc -g ex1buggy.c}.
The \texttt{-g} flag adds debug information. 
Running your program should yield a segmentation fault message.\\

Launch \texttt{gdb a.out}, then type \texttt{run} followed by the \texttt{ENTER} key. 
You should then see the error line.

\linerule

4. Copy \texttt{ex2.c} to \texttt{ex2buggy.c}, add a memory leak bug (e.g. by removing \texttt{free()}), and compile with \texttt{-g}. 
Then launch \texttt{valgrind --leak-check=full ./a.out}. 
This will show you that there is a memory leak in your program.



\end{document}