\documentclass[a4paper,11pt]{article}
\include{../../../latex_headers/lab_header}


% ---------------------------------- Document ----------------------------------
\begin{document}
\pagenumbering{arabic}

\begin{center}
{\Large CMPUT 350 Lab 9 Prep Problems}
\end{center}

% Add important dates, change TA, add TA tools

\linerule


\section*{Tic-Tac-Toe}
\begin{itemize}
    \item Played on a 3x3 board
    \item Two players move alternately by marking an empty square with their symbol
        (x and o, respectively)
    \item The player who first creates 3 of his symbols in a row (horizontally, vertically, or diagonally) wins
    \item The game ends when there are no moves left
    \item If at that time nobody has created a 3-in-a-row configuration, the game is counted as a draw
\end{itemize}

Sample Game:
\begin{verbatim}
    ---  x   --x   o    --x  x   -xx  o   -xx  x   xxx
    --- ---> ---  --->  -o- ---> -o- ---> -oo ---> -oo  x wins
    ---      ---        ---      ---      ---      ---
\end{verbatim}


\section*{MiniMax Search Applied to Tic-Tac-Toe}
For a review of \textit{MiniMax Search}, see the AI part 4 notes \textbf{part4.pdf}.

\subsection*{Game Tree}
\begin{itemize}
    \item Vertices represent game states
    \item The root vertex represents the state at the beginning of the game
    \item Edges represent moves
    \item Leaf vertices mark the end of a particular move sequence. 
        The game ends there (terminal game states)
    \item Each leaf has an associated payoff value (e.g., +1 for player MAX)
\end{itemize}

\subsection*{MiniMax Search}
\begin{itemize}
    \item MiniMax search backs up values starting with leaves
    \item Assumes we have two players: MAX and MIN (x and o in this case)
    \item Uses the minimax rule (maximizing scores in MAX nodes, minimizing in MIN
         nodes) if the values are stored in view of player MAX
    \item \textbf{Alternative}: NegaMax search always evaluates states in view of the player
    to move. This simplifies the code, because we only ever need to maximize
    scores. 
    \item \textbf{Important}: assuming alternating moves the returned value has to be
    negated
\end{itemize}

\subsection*{Tic-Tac-Toe Game Tree 1}
{\small
\begin{verbatim}
                 ---
                 ---                   [x to move]
                 ---
    
          / / / / | \ \ \ \            9 moves; pick one with highest value
                                       in view of player to move
        x--               ---
        ---     .....     ---          [o to move]
        ---               --x
      ////\\\\          ////\\\\       8 moves in each state
     
     xo-                       o--
     ---  ...         ...      ---     [x to move]
     ---                       --x
      .        etc.             .
      .         .               .
      .         .               .
     xox        .              oxo
     oox                       xox     [o to move]
     oxx                       oxx
    
     x wins                    o wins    
\end{verbatim}
}

\subsection*{Tic-Tac-Toe Game Tree 2}
What is the value of this game state in view of player x?

{\small
\begin{verbatim}
    xoo
    --x  [x to move]
    -o- 
\end{verbatim}
}
[ draw game tree on a piece of paper ]


\subsection*{Game Tree Search Mechanics}
MiniMax or NegaMax (preferred) search can be implemented by
depth-first-search, copying states and applying moves to the copied states as
we go down the game tree until we reach a terminal state, which gets evaluated
and its value passed back to the caller for maximization of the negated value
(NegaMax) or minimization/maximization of the returned value depending on
whose turn it is (MiniMax)

\medskip

Please refer to lecture notes \textbf{part4.pdf} 
(especially page 27) for details and code snippets you may want to use.



\subsection*{Tuples}
In this exercise you'll use \texttt{std::tuple}, which is a generalization
of \texttt{std::pair}. Tuples contain a sequence of data items that don't
necessarily share the same type. Individual items can be accessed by
\texttt{std::get<i>(tuple)}.

\medskip

\textbf{Example}:
{\small
\begin{verbatim}
    std::tuple<int,double,char> t(0, 3.5, 'x');
    std::cout << get<0>(t) << " " << get<1>(t) << " " << get<2>(t) << std::endl;
    // prints 0 3.5 x
\end{verbatim}
}


\linerule

1. In this problem you will implement missing functions of a tic-tac-toe program
that read a game state from a string, print it to stdout, and determine
whether a board is completely filled.\\

In file \texttt{ttt1.cpp}, which is given, implement the parts marked with

{\small
\begin{verbatim}
    // ... implement
\end{verbatim}
}

You may want to use this script file to compile your program: \texttt{make1}

\end{document}