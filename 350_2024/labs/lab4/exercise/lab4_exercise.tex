\documentclass[a4paper,11pt]{article}
\include{../../../latex_headers/lab_header}


% ---------------------------------- Document ----------------------------------
\begin{document}
\pagenumbering{arabic}

\begin{center}
{\Large CMPUT 350 Lab 4 Exercise Problems}
\end{center}

\labrules{g++ -Wall -Wextra -Wconversion -Wsign-conversion -O -g -std=c++17 ...}{Rational.h Rational.cpp main.cpp}{4}


1. [36 marks] In this exercise you will implement a rational number type that is meant as an
exact arithmetic replacement for rounding-error-prone floating point
types. Rational numbers are represented as two integer variables: numerator
\texttt{num} and denominator \texttt{den > 0}. The rational number they represent is num/den.

\begin{cppcode}
class Rational {
public:
    ...
private:
    // only data
    int num, den;
};
\end{cppcode}


At all times, \texttt{den} must be \texttt{> 0}. In this exercise we won't deal with arithmetic
overflow or simplifying rational numbers by dividing num,den by their greatest
common divisor. However, division by 0 must be checked with an assertion, and
\texttt{operator==} must work correctly, i.e. two rational numbers are equal if the
ratios represent the same value (e.g., \texttt{2/4 == 4/8}).

\pagebreak

Declare operators and methods in \texttt{Rational.h }and implement all of them in
\texttt{Rational.cpp }so that the given code in \texttt{main.cpp} works.

ALL operators and methods have to be implemented in \texttt{Rational.cpp}
File \texttt{main.cpp} must contain all test code AND code that triggers division by
0 and an illegal constructor call (division by 0),
\textbf{but you may leave that code commented out so that your code runs without 
crashing but that we can see that you have tested your code for those cases.}

\end{document}