\documentclass[a4paper,11pt]{article}
\include{../../../latex_headers/lab_header}


% ---------------------------------- Document ----------------------------------
\begin{document}
\pagenumbering{arabic}

\begin{center}
{\Large CMPUT 350 Lab 4 Prep Problems}
\end{center}

% Add important dates, change TA, add TA tools

\linerule

1. For class 
\begin{cppcode*}{linenos=false}
struct Point {
    int x{};
    int y{};
};
\end{cppcode*}
in \texttt{p1.cpp}, implement global operators \texttt{<<} and \texttt{>>}
that allows you to read and write points from \texttt{stdin}/\texttt{stdout} like so:
\begin{cppcode*}{linenos=false}
Point p;
std::cin >> p;
std::cout << p << std::endl;
\end{cppcode*}
\textbf{Note}: For this to work, you need to include \texttt{iostream}.

\linerule 

2. In file \texttt{p2.cpp}, 
implement the following class operators for class \texttt{Point} above:
\begin{center}
    \texttt{== ~~~~~~  > ~~~~~~  >=}
\end{center}
which work componentwise. 
I.e., two points are equal if their \texttt{x} components match and their \texttt{y} components match.
For \texttt{>} and \texttt{>=}, 
use the lexicographic ordering with \texttt{x} being the higher-valued component.

\linerule 

3. In file \texttt{p3.cpp},
implement \texttt{pre++}, \texttt{post++}, \texttt{pre--}, \texttt{post--},
for class \texttt{Point} above.
Their effect is to increment or decrement a point's \texttt{x} component, respectively.
Also, test your implementation.

\linerule 

4. Suppose you want to count how many \texttt{Point} instances have been constructed 
at any given time when your program runs. 
Write code in \texttt{p4.cpp} that gives you access to this information. 
Can your solution be compromised by a teammate working on a different project file?
If so, try to improve your design.

\end{document}