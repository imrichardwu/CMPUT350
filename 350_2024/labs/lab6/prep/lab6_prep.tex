\documentclass[a4paper,11pt]{article}
\include{../../../latex_headers/lab_header}


% ---------------------------------- Document ----------------------------------
\begin{document}
\pagenumbering{arabic}

\begin{center}
{\Large CMPUT 350 Lab 6 Prep Problems}
\end{center}

% Add important dates, change TA, add TA tools

\linerule

1. In file \texttt{pp1.cpp} implement class template \texttt{Queue} that supports the 
following standard first-in-first-out queue functions:
{\small 
\begin{verbatim}
    empty() const   : returns true iff queue is empty
    front()         : returns reference to front element (precond.: !empty())
    front() const   : returns const reference to front element (precond.: !empty())
    pop()           : removes front element (pre-cond.: !empty())
    push(x)         : add element x of type T at the end
\end{verbatim}
}

Use private inheritance from \texttt{std::list<T>} and delegate function calls to base
class function calls, like so:


\bigskip 
Examples:
\begin{cppcode*}{linenos=false}
template ...
class ...
    : private std::list<T>
{
    using Base = std::list<T>;
    ...
    bool empty() const {
        return Base::empty();
    }
    ...
}
\end{cppcode*}

Also make sure that your CC, AO, and destructor work (either by convincing
yourself that the default implementation works or by writing test code), and
check preconditions with assert. Also, test your code.

\linerule 

2. You have decided that for your nuclear power plant control software using
floating point variables is a bad idea because of potentially devestating
rounding errors. Your idea is to prevent users from instantiating your
class templates with floating point types with the help of type trait
class \texttt{is\_fp} and \texttt{static\_assert(cond, msg)} which checks a
Boolean condition at \textbf{COMPILE TIME} and prints an error msg if it is false.

\medskip 
Example:
\begin{cppcode*}{linenos=false}
template <typename T> 
class Foo {
    // don't allow T to be floating point type
    static_assert(...);
    ...
};
cout << is_fp<double>::value << endl;   // 1
cout << is_fp<float>::value << endl;    // 1
cout << is_fp<int>::value << endl;      // 0
Foo<int> f;         // ok
Foo<double> g;      // fails
\end{cppcode*}

In file \texttt{pp2.cpp} implement class template \texttt{is\_fp<T>} whose
variable value is 1 if \texttt{T} is a floating point type, and 0
otherwise. Test it using the code above.




\end{document}