\documentclass[a4paper,11pt]{article}
\include{../../../latex_headers/lab_header}


% ---------------------------------- Document ----------------------------------
\begin{document}
\pagenumbering{arabic}

\begin{center}
{\Large CMPUT 350 Lab 6 Exercise Problems}
\end{center}

\labrules{g++ -Wall -Wextra -Wconversion -Wsign-conversion -O -g -std=c++17 ...}{Stack.h mainStack.cpp mainSum.cpp}{6}


1. [43 marks] 
\begin{enumerate}[label=\alph*)]
\item In file \texttt{Stack.h} write a template class \texttt{Stack<T>} that stores elements
of type \texttt{T} and supports the following public class methods:

{\small
\begin{verbatim}
    empty  // returns true iff stack contains no element
    top    // returns reference to top element (pre-cond: !empty)
    push   // push an element on top of the stack
    pop    // remove the top element from stack (pre-cond: !empty)
    clear  // remove all elements
\end{verbatim}
}

In your implementation use private inheritance from \texttt{std::vector<T>}. All
Stack member functions except \texttt{clear()} must run in amortized constant time

\item In \texttt{Stack.h} implement
\begin{cppcode*}{fontsize=\footnotesize,linenos=false}
template <typename T>
std::ostream &operator<<(std::ostream &os, const Stack<T> &s);
\end{cppcode*}

that prints a stack to an output stream (in one line, starting with the top
element, using a single space (' ') as separator). You may assume that stack elements can be
printed using \texttt{cout << ...}

\medskip

Note that above \texttt{operator<<} can only use \texttt{Stack}'s public interface
(empty/top/push/pop/clear)

\item In \texttt{Stack.h} also write global template function "\texttt{reverse\_stack}" that
    reverses the order of elements in a given Stack.
    Think about how to do this using only the public interface.

\item In file mainStack.cpp write test code that checks all functions
    you wrote. The weight of correct implementation and testing is equal in
    this exercise.
\end{enumerate}

\bigskip

\textbf{IMPORTANT}! Even if you haven't implemented some functions, you can still write
test code for them!

\bigskip

Also check pre-conditions with assert, use const as often as possible, and
ensure that you can push expression results (e.g., \texttt{Stack<int> s; s.push(3+5);} )

\bigskip

Finally, make sure that your code doesn't leak memory!

\linerule 


\bigskip

2. [10 marks] 
In file \texttt{mainSum.cpp} write and test template class \texttt{Sum} that for $n \ge 0$
computes \texttt{1+2+..+(n-1)+n} at \textbf{COMPILE TIME}
\begin{cppcode*}{fontsize=\footnotesize,linenos=false}
int main() {
    cout << Sum<0>::value  << endl;         // 0
    cout << Sum<10>::value << endl;         // 55
    cout << Sum<20>::value << endl;         // 210
    cout << Sum<100>::value << endl;        // 5050
    // etc.
    return 0;
}
\end{cppcode*}

\end{document}