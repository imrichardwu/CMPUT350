\documentclass[a4paper,11pt]{article}
\include{../../../latex_headers/lab_header}


% ---------------------------------- Document ----------------------------------
\begin{document}
\pagenumbering{arabic}

\begin{center}
{\Large CMPUT 350 Lab 3 Exercise Problems}
\end{center}

\labrules{g++ -Wall -Wextra -Wconversion -Wsign-conversion -O -g -std=c++17 ...}{AnimalSim.cpp}{3}

[31 marks] In file \texttt{AnimalSim.cpp} create a class hierarchy for simulating animal behaviour in the wild. 
Each animal has an x,y location, an alive flag, an age, a print
function, and an act function which may change the animal's location, create
at most one new animal ("offspring"), or kill another animal. 

\medskip

The act function in the Animal base-class has the following signature:
   \[ \cppinline{virtual void act(World &w) = 0;} \]
where \texttt{w} is a \texttt{World} that holds an array of \texttt{Animal} pointers which describes the
current state of the simulation (see below).

\medskip

The print function prints animal data and has this signature:
\[ \cppinline{virtual void print() const = 0;} \]


\bigskip

1. Based on your Animal base-class, define
\begin{itemize}
    \item Mouse which always walks West  (decrements x)
    \item Goose which always flies North (increment y)
    \item Rabbit which creates one Rabbit each time it acts (at the same location)
    \item Bear which kills a random animal each time it acts. Bears can't kill
      themselves and can only choose targets among animals that existed at the
      beginning of the iteration (i.e. they can kill other bears, but not themselves)
\end{itemize}


2. Implement missing code parts (marked with "implement ...")

\bigskip 

3. To test your implementation use \cppinline{World::print()} which prints the current 
list of animals to \cppinline{stdout}.
In \cppinline{World::print()} a loop must call virtual Animal function \cppinline{print()} on each
animal in turn.

\medskip

The output format of \cppinline{World::print()} is as follows:
\begin{verbatim}
    <space> <Animal-Number> animal(s)
    <space> <Animal-Type> <Location(x-y)> <Age>
    <space> <Animal-Type> <Location(x-y)> <Age>
    ...      
\end{verbatim}

E.g. output of \texttt{simulate(1)} for a world containing 6 animals:
\begin{verbatim}
    iter 0
      6 animal(s)
      Mouse 100 100 0
      Mouse 50 60 0
      Goose 90 80 0
      Goose 10 20 0
      Rabbit 80 80 0
      Bear 70 90 0

    iter 1
      6 animal(s)
      Mouse 101 100 1
      Mouse 51 60 1
      Goose 90 79 1
      Rabbit 80 80 0
      Rabbit 80 80 1
      Bear 70 90 1
\end{verbatim}

Note that in this simulation the bear killed the second goose in the first
step. The order of animals is implementation-specific. What matters is that
all animals are accounted for.

Test your simulation on the starting configuration shown above with 20 steps.

\bigskip 

\textbf{Tips}:
\begin{itemize}
    \item Familiarize yourself with the public interface of class World
    \item Newly created animals don't act during the simulation step they were created in
    \item To create new offspring, \texttt{w.add\_animal(...)} needs to be called
    \item Use const as often as you can
    \item For functions that don't use their parameters use this construct
        \[ \cppinline{void act(World &) { ... }} \]
        (no parameter name) to avoid compilation warnings, 
        or alternatively,
        \[ \cppinline{void act([[maybe_unused]] World &w) { ... }} \]
        This is known as an \textit{attribute}, which is supported in {\CC}17 
        to indicate to the compiler that this variable may not be used,
        and thus to not warn about it.
    \item Use the given starting codebase \texttt{AnimalSim.cpp}, and implement and test the missing pieces,
        designated with \texttt{implement ...}
\end{itemize}


% \cppfile{Matrix.h}


\end{document}